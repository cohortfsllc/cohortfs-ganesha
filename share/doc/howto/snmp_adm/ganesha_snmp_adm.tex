%% LyX 1.3 created this file.  For more info, see http://www.lyx.org/.
%% Do not edit unless you really know what you are doing.
\documentclass[english]{article}
\usepackage[T1]{fontenc}
\usepackage[latin1]{inputenc}
\usepackage{graphicx}

\makeatletter
%%%%%%%%%%%%%%%%%%%%%%%%%%%%%% User specified LaTeX commands.

\usepackage{babel}
\makeatother
\begin{document}

\author{thomas.leibovici@cea.fr}
\title{\center{ \Large \textbf{Exporting GANESHA's statistics with SNMP} }}
\maketitle

%% \tableofcontents
%% \newpage

GANESHA is able to export its internal statistics so that an administrator can
browse them using the SNMP protocol. It also provides a client for easily
broswing its SNMP tree in a convivial and human understandable way.

This document describes the requirements for this feature (libraries needed,
SNMPd configuration...) and how to enable and configure it in GANESHA NFSD.

\section{Requirements}

SNMP support in GANESHA is based on the Net-SNMP library. This is a free
implementation of SNMP that comes with most Linux distributions. It can also be
retrieved from \texttt{http://net-snmp.sourceforge.net}.

GANESHA's SNMP support has been validated with Net-SNMP v5.1.4. However,
Net-SNMP v5.4 or higher is recommended.

Install the following packages on the machine where you are compiling and
running GANESHA NFSD:

\begin{itemize}
\item \emph{net-snmp} contains the snmpd and snmptrapd daemons ;
\item \emph{net-snmp-utils} contains various utilities for snmp ;
\item \emph{net-snmp-perl} contains the perl API ;
\item \emph{net-snmp-libs} XXXXXXXXXXXXXXXXXXXXXXXXXXXXXXXXXXXXXXXXXXXX ;
\item \emph{net-snmp-devel} contains the development libraries and
header files.
\end{itemize}

Note that the net-snmp library needs symbols defined in \emph{lm-sensors},
\emph{zlib}, and \emph{openssl} libraries, so you will also need to install
\emph{lm-sensors-devel}, \emph{zlib-devel}, and \emph{openssl-devel} packages on
your system.

\medskip
For using the SNMP client tool \texttt{snmp\_adm} provided with GANESHA, you
will also need the following perl modules:

\begin{itemize}
\item
  \emph{SNMP}\footnote{\texttt{http://search.cpan.org/~gsm/SNMP-5.0400001/SNMP.pm}}
  provided by the \emph{net-snmp-perl} RedHat's package
\item \emph{XML::DOM}\footnote{\texttt{http://search.cpan.org/~tjmather/XML-DOM-1.44/lib/XML/DOM.pm}}
  for XML outputs.
  \item
  \emph{Getopt::Std}\footnote{\texttt{http://search.cpan.org/~nwclark/perl-5.8.8/lib/Getopt/Std.pm}}
  to parse command line options.
  \item \emph{Config::General}\footnote{\texttt{http://search.cpan.org/~tlinden/Config-General-2.33/General.pm}}
 to parse config file
\item
  \emph{SNMP::Trapinfo}\footnote{\texttt{http://search.cpan.org/~tonvoon/SNMP-Trapinfo-1.0/lib/SNMP/Trapinfo.pm}}
  to manage SNMP traps.
\end{itemize}


\section{SNMPd setup}

GANESHA does not handle SNMP requests directly. Actually, it registers its SNMP sub-tree
on a SNMPd daemon, using an extention of the SNMP protocol called \emph{AgentX}.

The SNMPd daemon that exports GANESHA's SNMP tree can be located on the same host,
but it can also run on a remote management station.

\medskip
For activating agentX extension, add the following line
to your SNMPd configuration file (default location is \texttt{/etc/snmp/snmpd.conf}):
\begin{verbatim}
master agentx
\end{verbatim}

You must then specify a way to communicate with GANESHA:
\begin{itemize}
\item if it runs on the same host, you can use a socket file.
In this case, add the following line to SNMPd configuration file \texttt{/etc/snmp/snmpd.conf}:
\begin{verbatim}
AgentXSocket <path to the socket file>
 
E.g:
AgentXSocket /var/tmp/agentx/sock_file
\end{verbatim}
\item In any case (local or remote SNMPd), you can also use a standard
socket connection:
\begin{verbatim}
AgentXSocket <protocol>[:<network interface address>]:<port number>

E.g:
# listening on a single network interface
AgentXSocket tcp:192.168.0.42:761

# listening on all network interfaces
AgentXSocket tcp:761
\end{verbatim}

Note that the AgentX default port number is 705.
\end{itemize}

\medskip
Your SNMPd is now ready for exporting GANESHA statistics. Restart it after you
changed its configuration file.

\medskip
Note: if you want to restrict the access of GANESHA's SNMP subtree, refer to the
snmpd documentation about \emph{views}, \emph{groups},
\emph{SNMPv2 communities}, and \emph{SNMPv3 authentication}.

Figure \ref{examplesnmpdconf} shows an example SNMP configuration file that
should work for localhost access to SNMP.
\begin{figure}[h]
\begin{verbatim}
##       sec.name  source          community
com2sec  local     localhost       ganesha
com2sec  mynetwork 192.168.122.0/24    ganesha

##     group.name sec.model  sec.name
group  MyRWGroup  any        local
group  MyROGroup  any        mynetwork

##           incl/excl subtree                          mask
view all     included  .1                               80

##                context sec.model sec.level prefix read   write  notif
access MyROGroup  ""      any       noauth    0      all    none   none
access MyRWGroup  ""      any       noauth    0      all    all    all


# System contact information
syslocation Unknown (edit /etc/snmp/snmpd.conf)
syscontact Root <root@localhost> (configure /etc/snmp/snmp.local.conf)

# Added for support of bcm5820 cards.
pass .1.3.6.1.4.1.4413.4.1 /usr/bin/ucd5820stat

# AgentX support
master agentx
agentXSocket   tcp:localhost:705
agentXTimeout  5
agentXRetries  2
\end{verbatim}
\caption{Example snmpd.conf}
\label{examplesnmpdconf}
\end{figure}

\section{Enabling SNMP support in GANESHA}

\subsection{Compilation}

By default, GANESHA is compiled without SNMP support.
To enable this feature, add \texttt{--enable-snmp-adm} argument
to the \texttt{configure} command line. To be safe, \texttt{make clean} should
be run before rebuilding GANESHA.

\begin{verbatim}
E.g:
./configure --with-fsal=FUSE --enable-snmp-adm
\end{verbatim}

\subsection{Configuration}

A specific section of GANESHA's configuration file is dedicated
to SNMP options.
This block must be labelized with the \texttt{SNMP\_ADM} tag.

It must include the following parameters:

\begin{itemize}
\item \textbf{snmp\_agentx\_socket}: the socket file or the network interface
for communicating with SNMPd (as it appears in the SNMPd configuration file).
\item \textbf{product\_id}: this number must be unique for each instance of GANESHA
you are exporting with the same SNMPd.
\item \textbf{snmp\_adm\_log}: The log file for SNMP related logs.

\item \textbf{export\_cache\_stats}: indicates if cache stats are exported.
\item \textbf{export\_requests\_stats}: indicates if NFS requests stats are exported.
\item \textbf{export\_maps\_stats}: indicates if UID/GID map stats are exported.
\item \textbf{export\_buddy\_stats}: indicates if memory usage stats are exported.

\item \textbf{export\_nfs\_calls\_detail}: indicates if detailled stats about NFS calls are exported.
\item \textbf{export\_cache\_inode\_calls\_detail}: indicates if detailled stats about metadata cache calls are exported.
\item \textbf{export\_fsal\_calls\_detail}: indicates if detailled stats about filesystem calls are exported.
\end{itemize}


\begin{verbatim}
E.g:

SNMP_ADM
{
    snmp_agentx_socket = "tcp:localhost:761";
    product_id = 2;
    snmp_adm_log = "/var/log/ganesha/snmp_adm.log";

    export_cache_stats    = TRUE;
    export_requests_stats = TRUE;
    export_maps_stats     = FALSE;
    export_buddy_stats    = TRUE;

    export_nfs_calls_detail = FALSE;
    export_cache_inode_calls_detail = FALSE;
    export_fsal_calls_detail = FALSE;
}
\end{verbatim}

\section{Browsing/Accessing GANESHA's SNMP tree}

\subsection{Tree description}

The SNMP tree of GANESHA is located under this OID:
\begin{verbatim}
.iso.org.dod.internet.private.enterprise.cea.snmp-admin.<product_id>
\end{verbatim}
where \texttt{product\_id} is the value you specified in GANESHA's config file.
If some MIBs are missing, you can however access the tree with the numeric OID:
\begin{verbatim}
.1.3.6.1.4.1.12384.999.<product_id>
\end{verbatim}

\medskip

The product subtree is planned to be divided in tree parts:
\begin{itemize}
\item \texttt{<product\_root>.0} contains dynamic statistics of the NFSD;
\item \texttt{<product\_root>.1} contains configuration values;
\item \texttt{<product\_root>.2} contains special OIDs for executing administrative
actions on the daemon (flushing cache, ...).
\end{itemize}

At this time, only the first subtree (\texttt{<product\_root>.0}) is exported.

\medskip

GANESHA's SNMP tree is self-descriptive and it can be understood without any
specific MIB installed on your system.

Thus, each exported value \texttt{<i>} is described by the following OIDs:

\begin{itemize}
\item \texttt{<product\_root>.0.<i>.0} contains the name of the variable
\item \texttt{<product\_root>.0.<i>.1} contains the description of it
\item \texttt{<product\_root>.0.<i>.2.0} is the type of the variable\footnote{This is mainly unsed
for handling 64 bits integers. Indeed, as SNMP doesn't support them, we return them
as a string but this field will indicate they must be interpreted as integers.}.
\item \texttt{<product\_root>.0.<i>.2.1} contains the value.
\end{itemize}

This tree is represented in the figure \ref{figtree}.

\begin{figure}
  \center \includegraphics[scale=0.6]{snmp_tree.eps}
  \caption{GANESHA's SNMP tree}
  \label{figtree}
\end{figure}

\medskip

Example of \texttt{snmp\_walk} on a GANESHA variable:

\begin{verbatim}
$ snmpwalk -v 2c -c public localhost  .1.3.6.1.4.1.12384.999.2.0.4
SNMPv2-SMI::enterprises.12384.999.2.0.4.0 = STRING: "cache_nb_entries"
SNMPv2-SMI::enterprises.12384.999.2.0.4.1 = STRING: "number of entries in cache"
SNMPv2-SMI::enterprises.12384.999.2.0.4.2.0 = STRING: "INTEGER"
SNMPv2-SMI::enterprises.12384.999.2.0.4.2.1 = INTEGER: 51
\end{verbatim}

\subsection{Examining GANESHA's Logging Levels Through SNMP}

Every log message from GANESHA is labelled first by the component it belongs to
and second by the severity of the message. An explanation of each component and
severity level can be found in the GANESHA Logging document.

All components and their ids are shown in Figure \ref{componentoid}.
\begin{figure}
  \center
  \begin{tabular}{ | l | l |}
    \hline Log Component  & ID Number \\
    \hline ALL & 0 \\
    \hline LOG & 1 \\
    \hline MEMALLOC & 2 \\
    \hline STATES & 3 \\
    \hline MEMLEAKS & 4 \\
    \hline FSAL & 5 \\
    \hline NFSPROTO & 6 \\
    \hline NFSV4 & 7 \\
    \hline NFSV4\_PSEUDO & 8 \\
    \hline FILEHANDLE & 9 \\
    \hline NFS\_SHELL & 10 \\
    \hline DISPATCH & 11 \\
    \hline CACHE\_CONTENT & 12 \\
    \hline CACHE\_INODE & 13 \\
    \hline CACHE\_INODE\_GC & 14 \\
    \hline HASHTABLE & 15 \\
    \hline LRU & 16 \\
    \hline DUPREQ & 17 \\
    \hline RPCSEC\_GSS & 18 \\
    \hline INIT & 19 \\
    \hline MAIN & 20 \\
    \hline IDMAPPER & 21 \\
    \hline NFS\_READDIR & 22 \\
    \hline NFSV4\_LOCK & 23 \\
    \hline NFSV4\_XATTR & 24 \\
    \hline NFSV4\_REFERRAL & 25 \\
    \hline MEMCORRUPT & 26 \\
    \hline CONFIG & 27 \\
    \hline CLIENT\_ID\_COMPUTE & 28 \\
    \hline STDOUT & 29 \\
    \hline OPEN\_OWNER\_HASH & 30 \\
    \hline SESSIONS & 31 \\
    \hline PNFS & 32 \\
    \hline RPC\_CACHE & 33 \\
    \hline
  \end{tabular}
  \caption{GANESHA's Logging Component Categories and OID values}
  \label{componentoid}
\end{figure}

The debug levels and their order of severity are as follows:
\begin{itemize}
\item NIV\_NULL - no messages are printed.
\item NIV\_MAJ - only major messages are printed.
\item NIV\_CRIT - critical messages or higher are printed.
\item NIV\_EVENT - event messages or higher printed.
\item NIV\_DEBUG - debug messages or higher are printed.
\item NIV\_FULL\_DEBUG - extremely verbose debug messages or higher are printed.
\end{itemize}$\\$

Figure \ref{logleveloid} OID as it pertains to GANESHA's logging system , using
$.1.3.6.1.4.1.12384.999.1.1.10.2.1$ as an example.

\begin{figure}[h]
\center
  \begin{tabular}{ | l | l |}
    \hline $.1.3.6.1.4.1$ & .iso.org.dod.internet.private.enterprise or more simply
    enterprises, \\&e.g. enterprises$.12384.999.1.1.10.2.1$ \\
    \hline  $.12384$ & CEA - the organization that originated Ganesha \\
    \hline  $.999$ & snmp\_adm - the program \\
    \hline  $.1$ & Product\_Id from the config file \\
    \hline  $.1$ & LOG\_OID - the portion of GANESHA's snmp tree devoted to logging \\
    \hline  $.0$ & component ID (all)\\
    \hline  $.2$ & logging level of the component \\
    \hline  $.1$ & specifies the \emph{value} of the logging level \\
    \hline
  \end{tabular}
  \caption{Deconstruction of an OID string that identifies the logging level of a
    component}
  \label{logleveloid}
\end{figure}

The following commands are example uses of the \emph{snmpwalk} command to query
for the log level of different components.$\\$$\\$
To see all information pertaining to the logging snmp variables:
\begin{verbatim}
$ snmpwalk -Os -c ganesha -v 1 localhost .1.3.6.1.4.1.12384.999.1.1
\end{verbatim}
To see just what is actually useful and fits in a reasonable window:
\begin{verbatim}
$ snmpwalk -Os -c ganesha -v 1 localhost .1.3.6.1.4.1.12384.999.1.1 | grep -v
'\"STRING\"' | grep -v ``Log level''
\end{verbatim}
To see logging information on a particular component:
\begin{verbatim}
$ snmpwalk -Os -c ganesha -v 1 localhost .1.3.6.1.4.1.12384.999.1.1.3 | grep -v
'\"STRING\"' | grep -v ``Log level''
\end{verbatim}

\subsection{Changing GANESHA's Logging Levels Through SNMP}

The following commands are example uses of the \emph{snmpset} command to change
the severity level of messages that should be logged for one or all
components.$\\$$\\$
To change the level for \emph{COMPONENT\_DISPATCH} (note that you specify the
level by name, just like in the config file or on the command line):
\begin{verbatim}
$ snmpset -Os -c ganesha -v 1 localhost .1.3.6.1.4.1.12384.999.1.1.10.2.1 s
NIV_FULL_DEBUG
\end{verbatim}
To change the level for all components:
\begin{verbatim}
$ snmpset -Os -c ganesha -v 1 localhost .1.3.6.1.4.1.12384.999.1.1.0.2.1 s
NIV_FULL_DEBUG
\end{verbatim}

\subsection{Using the SNMP client provided with GANESHA}

Even if GANESHA statistics can be browsed using standard SNMP commands
(\texttt{snmp\_get}, \texttt{snmp\_walk}...), GANESHA comes with a SNMP client tool
(\texttt{snmp\_adm})
for easily browsing those stats in a convivial way, without having to handle OIDs,
and all SNMP relative stuff.

It is located in the '\texttt{snmp\_adm/client}' directory of GANESHA's distribution,
and is also available in GANESHA RPMs that include SNMP support.

\subsubsection{\texttt{snmp\_adm} client configuration file}

If you're bored of typing OIDs, SNMP version, community name and all that stuff,
just create a \texttt{.snmp\_adm.conf} file in your home (with mode 600) with the following
lines inside:
\begin{verbatim}
host       <snmpd_address>[:<snmpd_port>]
product_id <the product id of your favorite NFS server>

#if you are using SNMPv3 protocol, also specify the following information:
# password for authentication
auth_pass "password"
# password for encoding
enc_pass "password"
# user name
sec_name "snmpadm"
\end{verbatim}

\subsubsection{SNMP relative options}

If you don't want to use a configuration file,
or if you want to overwrite the values it specifies,
you can indicate them on \texttt{snmp\_adm} command line:
\begin{verbatim}
SNMP relative options:

  -s <host>[:port] : the host where SNMP server is running
      (default is localhost)
  -p <product_id|product_name> : the deamon to be queried
      (default is the first product of server's admin tree)
  -C <community>: Community name for SNMPv2c (default is public).
  -A <auth>: authentication for SNMPv3.
  -X <pass>: password for SNMPv3.
  -u <secname>: security name for SNMPv3.
  -f <path>: path to the configuration file.
\end{verbatim}

\subsubsection{\texttt{snmp\_adm} commands}

The main command you will use is '\texttt{snmp\_adm liststat}'.
When used without options, it only displays the list of available variables.
When used with '\texttt{-d}' it displays the description of each variable.
When used with '\texttt{-v}' it displays the values of variables.
You can also specify an expression, so only the variables
whose name contains this expression will be displayed.

\begin{verbatim}
E.g:

$ ./snmp_adm -v liststat cache

Statistics for product_id=2:
name                          type       value

cache_nb_gc_lru_active         INTEGER    176
cache_nb_gc_lru_total          INTEGER    432
cache_nb_call_total            INTEGER    25323
cache_nb_entries               INTEGER    5176
cache_min_rbt_num_node         INTEGER    32
cache_max_rbt_num_node         INTEGER    37
cache_avg_rbt_num_node         INTEGER    34
cache_nbset                    INTEGER    5465
cache_nbtest                   INTEGER    0
cache_nbget                    INTEGER    6243
cache_nbdel                    INTEGER    132
\end{verbatim}

\section{Troubleshooting}

Make sure the agentX addresses are the same in both snmpd.conf and the ganesha
configuration file:
\begin{verbatim}
$ grep Snmp_Agentx_Socket /etc/ganesha/gpfs.ganesha.main.conf
Snmp_Agentx_Socket = "tcp:localhost:705" ;
$ grep agentXSocket /etc/snmp/snmpd.conf
agentXSocket   tcp:localhost:705
\end{verbatim}
Iptables could potenetially block snmp. The default iptables setup for RHEL 5
should be fine for local use of SNMP.
If GANESHA was compiled with SNMP support the following message will be logged
soon after startup with NIV\_EVENT:
\begin{verbatim}
$ grep "SNMP stats service was started successfully" /var/log/messages
Sep  8 15:14:46 localhost nfs-ganesha[25074]: [stat_snmp] :MAIN: EVENT: NFS \
STATS: SNMP stats service was started successfully
\end{verbatim}
Make sure SNMP and GANESHA are running. In the following example we are using
the GPFS FSAL. An alternative FSAL will have a different name for the init
script.
\begin{verbatim}
$ /etc/init.d/snmpd status
snmpd (pid  24991) is running...
$ /etc/init.d/nfs-ganesha.gpfs status
GPFS Ganesha is running.
Open-by-handle module is loaded.
\end{verbatim}
%To check if SNMP encountered errors:
%\begin{verbatim}
%
%\end{verbatim}


\end{document}


